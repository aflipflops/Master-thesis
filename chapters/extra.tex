\chapter{extra}
Before a chemical reaction takes place at the surface, reactant(s) must be transported from the bulk solution onto the reacting surface, or product(s) moved away from the surface to make space for further reactions. Thus, a reaction becomes transport controlled/limited when the movement of reactants or products is slower than the surface reactions.

As an example, dissolution of calcite shows transport control at low pH and becomes interface limited at higher pH (see Figure 5.4 in \cite{Brantley2008b}. \textbf{ [Provide figure for understanding]}

The chemical weathering of rocks at Earth's surface controls soil fertility, ocean salinity, transport and sequestration of contaminants, cycling of metals in the environment, and formation of ore deposits \citep{hellmann2012}.
The knowledge of reaction control is important because it helps us to calculate the rate of the reaction or rather what is the correct way to calculate rate. For e.g., if we are certain that rate is interface limited, then the rate is normalised to surface area and expressed in \citep{Brantley2008c}. \SI{45}{years} after the publication of the first study by \cite{luce1972} on forsterite dissolution, we now have robust information on reaction rate and mechanism (See \cite{rimstidt2012} for compilation of reaction rate studies in laboratory environment). However, finding weathering rates at field scale have consistently produced values several order of magnitudes slower than those determined in laboratory experiments (\cite{white1995} referenced in \citep{renforth2015}. This naturally poses a problem while calculating the carbon sequestration potential of enhanced weathering at a global scale.  

The solid earth comprises of rocks formed either at the earth's surface through sedimentary processes (sedimentary), cooling of molten rock brought to the surface (igneous), and reformation of previously formed igneous or sedimentary rocks under extreme stress (metamorphic). Rocks are further composed of minerals - an ordered arrangement of elements (crystal), with a definite chemical composition. For e.g., the rock granite consists of feldspar, quartz, mica, and amphibole - all minerals.

but for the case of olivine dissolution in open systems at ambient conditions, \cref{forsterite,rate} become respectively (modified from Supporting online material,\cite{rimstidt2012}):
\begin{equation} \label{eq:olivine_diss}
(Mg_{x}Fe_{1-x})_{2}SiO_{4}+4H_2O+4CO_2\longrightarrow 2xMg^{+2} +2(1-x)Fe^{+2} + 4HCO_3^- + H_{4}SiO_4 
\end{equation}
and,
\begin{equation} \label{eq:olivine_rate}
r=-\frac{dn_{fo}}{dt}=\frac{1}{2x}\frac{d[Mg]^{+2}}{dt}=\frac{1}{4}\frac{d[HCO_3^-]}{dt}=\frac{d[H_{4}SiO_4]}{dt}
\end{equation}
here $dn_{fo}/dt$ is the rate of consumption of forsterite normalised again to surface area. 


\cref{sa_sec} contains two important points. First, while the reaction rate depends on the reactive surface area, current methods can only allow robust measurement of BET/Total external surface area and the geometric surface area. One might then argue that if most of the previous experiments used either of the two methods, why could it be of some significance in explaining the lower reaction rate? The reason lies in the sample pre-treatment before BET measurements and the reactor type involved. The pre-treatment of the sample often involves dissolution in alcohol over-night followed by ultrasonication. These processes remove ultra-fines and 'smooths out' sites of high surface energy on the mineral. The samples in the current experiment suffered no chemical pre-treatment, and thus show power-law kinetics, also discussed in \cref{nonsteadystate}.  Reaction rates are mostly normalised to initial BET surface area. The change in surface area 

The dissolution of olivine in open systems and ambient conditions follows the reaction shown in \cref{results_olivine} (modified from  \cite{rimstidt2012}), and the corresponding rate equation in terms of consumption of reactants and release of products is given in \ref{results_for}. The theory of reaction rate was covered in \cref{rate_control}.
\begin{equation} \label{results_olivine}
(Mg_{x}Fe_{1-x})_{2}SiO_{4}+4H_2O+4CO_2\longrightarrow 2xMg^{+2} +2(1-x)Fe^{+2} + 4HCO_3^- + H_4SiO_4 
\end{equation}

\begin{equation}
\label{eq:results_for}
r=-\frac{dn_{fo}}{dt}=\frac{1}{2x}\frac{d[Mg]^{+2}}{dt}=\frac{1}{4}\frac{d[HCO_3^-]}{dt}=\frac{d[H_{4}SiO_4]}{dt}
\end{equation}
Here, $dn_{fo}/dt$, is the rate of consumption of forsterite normalised to surface area.The current experimental setup is a batch-type column reactor, and reaction rate can be calculated for each `batch addition´ of water, i.e., for each day water is added to the column and left for a day to flow through.


\section{Mechanism of reaction}
The non-stochiometric dissolution of forsterite and observance of an amorphous Si layer, has led to two main theories on mechanism of forsterite dissolution. Leached layer mechanism- Forsterite dissolution mechanism are either- i) exchange reaction between H+ and Mg \citep{pokrovsky2000} via solid-state volume inter-diffusion. 
ii) Chemical hydrolysis reaction releases Si and O into the bulk solution at the outer surface. Thus, the mineral rate retreat is controlled by i) and rate of retreat of external surface is controlled by ii). Once the two rates become equal, steady state is achieved. 


The rate of evaporation in laboratory conditions was calculated to be around \SI{3}{\milli\litre\per\day}. 

This simulates a rainfall of \SI{\sim 11000}{mm.y^{-1}}, a maximum observed in tropical forests which are potential regions of application for EW. 

A general mineral dissolution reaction is described by 
\begin{equation} \label{eq:min}
\ce{(A_aB_b)(s) -> aA^{qA}(aq) + bB^{qB}(aq)}
\end{equation}
The rate of a reaction is defined as the rate at which a solid mineral reactant(s) is consumed or aqueous product(s) formed. The steady-state rate for equation \ref{eq:min} (expressed in mol mineral reacted per unit time per unit volume) is described by:
\begin{equation}\label{eq:rate}
r=-\frac{d[A_aB_b]}{dt}=\frac{\text{1}}{a}\frac{d[A_{(aq)}]}{dt}=\frac{\text{1}}{b}\frac{d[B_{(aq)}]}{dt}
\end{equation}
where, $[A]$ refers to the concentration of species, $a$ and $b$ are the stochiometric coefficients, $t$ is time, and $q$ is the charge of the aqueous species \citep{Brantley2008b}. 
Reactions occurring at the mineral/fluid boundary are often shown to be dependent on the total or reactive surface area, and hence, rates are normalised to it. 

The factors affecting the rate of mineral dissolution can be summarised into one equation \citep{renforth2015}:
\begin{equation}
R= A \times k \times(\alpha_a)^{n}\times(\text{1}-\omega)
\end{equation}
where $A$ is the surface area of the mineral taking part in the reaction, also called reactive surface area (\si{\square\metre}), $k$ is the reaction constant (units depend on the rate law), $\alpha_a$ is the activity of the reactant species in the fluid (\si{\mole\per\litre}), and  $\omega$ is the saturation index of the dissolving species, and is dimensionless.

\noindent The \textbf{reaction constant},$k$, is found by the Arrhenius equation and depends on the mineral composition and temperature of the system:
\begin{equation}
k=e^{-E_a/RT}
\end{equation}
$E_a$ is the activation energy (\si{\joule\per\mole}) of the rate determining step, $T$ is the temperature (K), and $R$ is the gas constant (\si{\joule\per\kelvin\per\mole}). 

\noindent \textbf{$(\alpha_{a})^{n}$} refers to the\textbf{ activity} of the reactant specie(s) or a molecular-complex and $n$ is often found out experimentally.

\noindent The \textbf{saturation index}, $\omega$, is the log ratio of - the product of the reactant activities to the solubility product of the ions. Therefore, $(1-\omega)$ is a measure of how far the system is from equilibrium. When $\omega=1$, the system is in equilibrium, $\omega<1$, the reactants are undersaturated and the reaction will go towards the products, and for $\omega>1$, the reactants are supersaturated and the reaction will tend towards the reactants \citep{schott2009}.

\section{The model of Rimstidt}
The curve of $J_R$ is obtained by fitting a rate equation into past data, and is \SI{\sim e-10.344}{\mole\per\metre\square\per\second}. $J_D$ calculated from model results, is \SI{e-13.4}{\mole\per\metre\square\per\second} for coarse and \SI{e-13.8}{\mole\per\metre\square\per\second} for fine, taken at \SI{25}{\degreeCelsius}.

The easiest and common way to finding surface reaction rates is by observing the change in concentration of the reactants or products with time \citep{rimstidt2012,Brantley2008b} though other methods also exist - \textit{surface retreat} \citep{awad2000} and \textit{shrinking particle model} 		   \citep{hanchen2006,prigiobbe2009}. They are, however, not used or discussed here. 

Long-term experiments (> 6 years) on granites revealed that the reaction of freshly crushed rock never reached steady state \citep{white2003}; favoring the viewpoint that steady state kinetics may never be achieved since the mineral continuosly changes during dissolution (\cite{white2003}; \cite{kohler2005} in \cite{Brantley2008b}).

There is a large range of distribution of BET area for some data points, e.g. \SI{100}{\micro\meter}. When performed for a similar sample in the same laboratory, BET measurements in general are robust (error of \SI{\sim \pm 5}{\percent} for SSA \SI{>0.4}{\square\metre\per\gram} and \SI{\sim \pm 70}{\percent} for SSA \SI{<0.1}{\metre\square\per\gram} \citep{brantley2000}), but across laboratories, different method of sample preparation and pre-treatment, age of mineral, adsorbent gas used \citep{brantley2000,rosso2000}.

\footnote{Geometric surface rates are on an average 5.2 times faster than BET, and thus, to compare rates obtained from the two methods geometric areas are often multiplied by a roughness factor (\num{\sim 5} for forsterite) \citep{rimstidt2012}}

\begin{mdframed}[leftmargin=10pt,rightmargin=10pt]
Calculating the lifetime of a particle using the model of \cite{lasaga2014}.
\begin{align}
t_l=\frac{D_o}{2kV_m}
\end{align}
$t_l$ is the lifetime of the particle, $D_o$ is the particle diameter, k is the dissolution flux or rate of reaction, and $V_m$ is the molar volume. Assuming a dissolution flux at pH=8 as \SI{e-10}{mol.m^{2}.s^{-1}}, molar volume  of \SI{4.365e-5}{m^{3}.mol^{-1}}, and particle diameter as \SI{20}{\micro\meter}.
\begin{equation}
t_l=\frac{D_o}{2kV_m}=\frac{\SI{1e-5}{m}}{2\left(\SI{e-10}{mol.m^{2}.s^{-1}}\right)\left(\SI{4.365e-5}{m^{3}.mol^{-1}}\right)}=\SI{1.14e9}{\second}
\label{eq:lifetime}
\end{equation}

This translates to \SI{36}{years}. 
\end{mdframed}


If a dissolution reaction is assumed to be surface-controlled than normalisation of the rate by the surface area should produce identical values irrespective of the particle size. This is based on the assumption that reactive sites are uniformly distributed on the mineral surface, a property that is conserved irrespective of the grain size. In other words, the relation between surface area and reaction rate is grain size independent. However, \cite{holdren1987}, experimenting with feldspars, show that this assumption breaks down for particles \SI{<75}{\micro\meter} for whom no clear relation exists between rate and surface area. In fact, in some samples, the smallest size fractions gave the least rate (normalised by weight) even though the specific surface area was orders of magnitude greater. According to \cite{kleiv2006}, it is impossible to predict the reactivity of activated olivine based solely on the BET surface area, i.e., no simple relation exists between BET area and reaction rate. Thus, it seems that normalisation by BET surface area is not the best measure to compare reaction rates across grain sizes when spread across a big range, as in the current experiment. 