\chapter{Conclusions, Implications, and Future Work}

To assess the potential of enhanced weathering as a method to sequester \ce{CO_2}, knowledge of the reaction rate of a silicate, e.g., forsterite, is needed. Past experiments on forsterite dissolution might not be suitable to answer this due to the conditions and setup under which they were performed. The current experiment fulfills this gap by undertaking long term packed-bed column experiments, open to atmosphere and without sample pre-treatment. The columns are separately filled with three different size fractions - coarse, fine, and mixed grain-type, and water is poured over them.

After running for almost one year results show that the olivine hasn't reached steady-state, which might be due to the presence of a large amount of fines in the column. The rate differs across the measured species -  $R_{Alk} \sim R_{Mg}$ > $R_{Si}$ due to reprecipitation of released Si as a Si-rich surface layer. The measured rate is many orders of magnitude lower than observed in past literature, and coarse grain-type having higher rate than fine and mixed. These results cannot be understood when rate is surface-limited, normally assumed in forsterite dissolution. The experiment results are further compared with a theoretical rate model for packed-bed. The model-results show transport controlling the reaction rate; which are lower than observed for Mg but in agreement for Si. This could be explained assuming the rate to be surface-controlled, thus faster, in the first few layers followed by transport-control in the rest of the bed. In conclusion, results from past laboratory experiments cannot be directly used to estimate \ce{CO_2} absorptive potential or to manipulate rate for EW application; because the system could become transport-controlled. 

Weathering rates measured in the laboratory have been found to be orders of magnitude greater than those observed in the field. The results from this study show that this could at least partly be explained based on transport-control; as column experiments better mimic natural conditions.

Other processes could influence olivine dissolution rates in nature, for e.g., the high p\ce{CO2} in soils through soil respiration processes, presence of complex ligands, role of soil bacteria etc. Even though the current experiment doesn't consider them, it puts forth a step in the right direction; of estimating field weathering rates and its influencing processes. Lastly, surface area is an important property affecting rate, and BET surface area shouldn't be substituted for the `actual and available surface area'; for it hinders rate comparability  across experiments and introduces sources of uncertainty. A novel method must be found out to move past this problem.

